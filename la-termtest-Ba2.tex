% la-termtest-Ba2.tex

\documentclass[11pt]{article}
\usepackage{alltt}
\usepackage{enumerate}
\usepackage{syllogism} 
\usepackage{october}
\usepackage[table]{xcolor}
\pagestyle{empty}

\newcounter{aufg}
\setcounter{aufg}{0}
\newcommand{\aufgabe}[1]{\refstepcounter{aufg}\textbf{(\arabic{aufg})} [#1 points]}

\begin{document}

\textbf{Term Test Ba version 2}

\aufgabe{5} Solve the following system of linear equations.
\begin{equation}
\begin{array}{ccccc}
4x&-&y&=&-8\\
-2x&+&\frac{1}{2}y&=&4
\end{array}\notag
\end{equation}

Provide the solution set in the following form, specifying $m,n,p,q$. If your solution is unique, then $p=0,q=0$.
\begin{equation}
\left\{\left(\begin{array}{c}x\\y\end{array}\right)\in\mathbb{R}^{2}\;\left\vert\;\left(\begin{array}{c}x\\y\end{array}\right)=\left(\begin{array}{c}m\\n\end{array}\right)+s\left(\begin{array}{c}p\\q\end{array}\right),s\in\mathbb{R}\right.\right\}\notag
\end{equation}

\aufgabe{5} Consider the vector space of 2x2 matrices. Are the following four matrices a basis for this vector space?
\begin{equation}
A=\left[
\begin{array}{cc}
8 & -10 \\
-3 & -9
\end{array}\right],B=\left[
\begin{array}{cc}
-5 & 0 \\
-4 & 1
\end{array}\right],C=\left[
\begin{array}{cc}
2 & -1 \\
-7 & -16
\end{array}\right],D=\left[
\begin{array}{cc}
1 & -9 \\
 0 & 8
\end{array}\right]\notag
\end{equation}
\begin{itemize}
\item If yes, find the coordinates in terms of this basis for 
\begin{equation}
E=\left[
\begin{array}{cc}
4 & -9 \\
-3 & 7
\end{array}\right]\notag
\end{equation}

\item If no, express one of the four given matrices by the other three.
\end{itemize}

\aufgabe{5} Consider the following three vectors in $\mathbb{R}^{3}$,
\begin{equation}
\left(
\begin{array}{c}
-8\\
-10\\
2

\end{array}\right),\left(
\begin{array}{c}
0\\
-1\\
-3

\end{array}\right),\left(
\begin{array}{c}
-2\\
6\\
5

\end{array}\right)\notag
\end{equation}
 Determine the three lengths of these vectors and the three angles between them in degrees (not radians). If they replace the origin to the points $P,Q,R$, determine the plane equation for the plane containing the three points, using the cross product.

\end{document}
