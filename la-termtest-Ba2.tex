% la-termtest-Ba2.tex

\documentclass[11pt]{article}
\usepackage{alltt}
\usepackage{enumerate}
\usepackage{syllogism} 
\usepackage{october}
\usepackage[table]{xcolor}
\pagestyle{empty}

\newcounter{aufg}
\setcounter{aufg}{0}
\newcommand{\aufgabe}[1]{\refstepcounter{aufg}\textbf{(\arabic{aufg})} [#1 points]}

\begin{document}

\textbf{Term Test Ba version 2}

\aufgabe{5} Solve the following system of linear equations.
\begin{equation}
\begin{array}{ccccccc}
4x&-&y&+&2z&=&-8\\
-2x&+&3y&+&7z&=&17\\
8x&+&3y&+&20z&=&10

\end{array}\notag
\end{equation}
If the system is consistent and dependent, provide your answer in the form
\begin{equation}
S=\{u\in\mathbb{R}^{3}\;|\;u\mbox{ corresponds to }\vec{u}=\vec{v_{0}}+s_{1}\vec{v_{1}}+{\ldots}+s_{n}\vec{v_{n}}\}\notag
\end{equation}
where $n$ is the dimension of the solution space and $s_{i}\in\mathbb{R}$ for $i=1,{\ldots},n$. Note that
$(-2,2,1)^{\intercal}$ solves the system.

\aufgabe{5} Consider the vector space of parabolas with the equation
$y=a(x-h)^{2}+k$. Are the following three parabolas a basis for this
vector space?
\begin{equation}
  \label{eq:iozooroo}
  \begin{array}{ccccc}
y&=&-6(x-5)^{2}&-&8\\y&=&(x-2)^{2}&-&4\\y&=&-15(x-4)^{2}&-&4
  \end{array}\notag
\end{equation}
\begin{itemize}
\item If yes, find the coordinates in terms of this basis for $y=-3(x-20)^{2}-66$.
\item If no, express one of the three given parabolas by the other two.
\end{itemize}

\aufgabe{5} Consider the following three vectors in $\mathbb{R}^{3}$,
\begin{equation}
\left(
\begin{array}{c}
-8\\
-10\\
2

\end{array}\right),\left(
\begin{array}{c}
0\\
-1\\
-3

\end{array}\right),\left(
\begin{array}{c}
-2\\
6\\
5

\end{array}\right)\notag
\end{equation}
 Determine the three lengths of these vectors and the three angles between them. If they replace the origin to the points $P,Q,R$, determine the plane equation for the plane containing the three points, using the cross product.

\end{document}
