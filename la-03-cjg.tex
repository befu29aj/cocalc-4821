% la-03-cjg.tex

\documentclass[xcolor=dvipsnames]{beamer}
\usepackage{teachbeamer}

\title{Cramer, Jordan, and Gauss}
\subtitle{{\CourseNumber}, BCIT}

\author{\CourseName}

\date{September 24, 2018}

\begin{document}

\begin{frame}
  \titlepage
\end{frame}

\begin{frame}
  \frametitle{Topics Covered}
  This lesson covers the following topics:
  \begin{enumerate}
  \item Cramer's Rule
  \item Echelon Forms and Gauss-Jordan Elimination
  \item Least Squares Approximation
  \end{enumerate}
\end{frame}

\begin{frame}
  \frametitle{Cramer's Rule}
  Cramer's rule makes finding the solutions to systems of linear
  equations very simple, at the expense of understanding what's going
  on. It's black magic. Consider the following system of linear
  equations:
  \begin{equation}
    \label{eq:keiyikae}
    \left[
      \begin{array}{ccc}
        3 & 4 & -2 \\
        1 & 2 & -1 \\
        3 & -1 & -3
      \end{array}\right]\cdot\left[
      \begin{array}{c}
        x \\
        y \\
        z
      \end{array}\right]=\left[
      \begin{array}{c}
        20 \\
        9 \\
        6
      \end{array}\right]
  \end{equation}
  The inverse of the coefficient matrix is
  \begin{equation}
    \label{eq:phiedier}
    -\frac{1}{7}\cdot\left[
      \begin{array}{ccc}
        -7 & 14 & 0 \\
        0 & -3 & 1 \\
        -7 & 15 & 2
      \end{array}\right]
  \end{equation}
  Therefore, the solution to the system is
  \begin{equation}
    \label{eq:suchociw}
    \left[
      \begin{array}{c}
        x \\
        y \\
        z
      \end{array}\right]=-\frac{1}{7}\cdot\left[
      \begin{array}{ccc}
        -7 & 14 & 0 \\
        0 & -3 & 1 \\
        -7 & 15 & 2
      \end{array}\right]\cdot\left[
      \begin{array}{c}
        20 \\
        9 \\
        6
      \end{array}\right]=\left[
      \begin{array}{c}
        2 \\
        3 \\
        -1
      \end{array}\right]
  \end{equation}
\end{frame}

\begin{frame}
  \frametitle{End of Lesson}
Next Lesson: Vectors
\end{frame}

\end{document}

