% la-03-cjg.tex

\documentclass[xcolor=dvipsnames]{beamer}
\usepackage{teachbeamer}

\title{Cramer, Jordan, and Gauss}
\subtitle{{\CourseNumber}, BCIT}

\author{\CourseName}

\date{September 24, 2018}

\begin{document}

\begin{frame}
  \titlepage
\end{frame}

\begin{frame}
  \frametitle{Topics Covered}
  This lesson covers the following topics:
  \begin{enumerate}
  \item Cramer's Rule
  \item Echelon Forms and Gauss-Jordan Elimination
  \item Least Squares Approximation
  \end{enumerate}
\end{frame}

\begin{frame}
  \frametitle{Cramer's Rule}
  Cramer's rule makes finding the solutions to systems of linear
  equations very simple, at the expense of understanding what's going
  on. It's black magic. Consider the following system of linear
  equations:
  \begin{equation}
    \label{eq:keiyikae}
    \left[
      \begin{array}{ccc}
        3 & 4 & -2 \\
        1 & 2 & -1 \\
        3 & -1 & -3
      \end{array}\right]\cdot\left[
      \begin{array}{c}
        x \\
        y \\
        z
      \end{array}\right]=\left[
      \begin{array}{c}
        20 \\
        9 \\
        6
      \end{array}\right]
  \end{equation}
  The inverse of the coefficient matrix is
  \begin{equation}
    \label{eq:phiedier}
    -\frac{1}{7}\cdot\left[
      \begin{array}{ccc}
        -7 & 14 & 0 \\
        0 & -3 & 1 \\
        -7 & 15 & 2
      \end{array}\right]
  \end{equation}
  Therefore, the solution to the system is
  \begin{equation}
    \label{eq:suchociw}
    \left[
      \begin{array}{c}
        x \\
        y \\
        z
      \end{array}\right]=-\frac{1}{7}\cdot\left[
      \begin{array}{ccc}
        -7 & 14 & 0 \\
        0 & -3 & 1 \\
        -7 & 15 & 2
      \end{array}\right]\cdot\left[
      \begin{array}{c}
        20 \\
        9 \\
        6
      \end{array}\right]=\left[
      \begin{array}{c}
        2 \\
        3 \\
        -1
      \end{array}\right]
  \end{equation}
\end{frame}

\begin{frame}
  \frametitle{Cramer's Rule}
  Finding the inverse, however, is time-consuming. Also, it gives us
  all three solutions, and we may only want the value of one of the
  variables and not all of them. Cramer's rule tells us that, for
  example,
  \begin{equation}
    \label{eq:aquohkol}
    y=\frac{\det(A_{y})}{\det(A)}
  \end{equation}
  where $A$ is the coefficient matrix and $A_{y}$ is the coefficient
  matrix with the second column (corresponding to $y$) replaced by the
  vector of constants. In other words,
\begin{equation}
  \label{eq:veefoubu}
  y=\frac{\det\left(\left[
        \begin{array}{ccc}
          3 & 20 & -2 \\
          1 & 9 & -1 \\
          3 & 6 & -3
        \end{array}
\right]\right)}{\det\left(\left[
      \begin{array}{ccc}
        3 & 4 & -2 \\
        1 & 2 & -1 \\
        3 & -1 & -3
      \end{array}\right]\right)}=\frac{-21}{-7}=3
\end{equation}
\end{frame}

\begin{frame}
  \frametitle{Cramer's Rule}
  For the other two variables,
  \begin{equation}
    \label{eq:sheethei}
    x=\frac{\det\left(\left[
        \begin{array}{ccc}
        20 & 4 & -2 \\
        9 & 2 & -1 \\
        6 & -1 & -3
        \end{array}
\right]\right)}{\det\left(\left[
      \begin{array}{ccc}
        3 & 4 & -2 \\
        1 & 2 & -1 \\
        3 & -1 & -3
      \end{array}\right]\right)}=\frac{-14}{-7}=2
  \end{equation}
  \begin{equation}
    \label{eq:ijohmahv}
    y=\frac{\det\left(\left[
        \begin{array}{ccc}
        3 & 4 & 20 \\
        1 & 2 & 9 \\
        3 & -1 & 6
        \end{array}
\right]\right)}{\det\left(\left[
      \begin{array}{ccc}
        3 & 4 & -2 \\
        1 & 2 & -1 \\
        3 & -1 & -3
      \end{array}\right]\right)}=\frac{7}{-7}=-1
  \end{equation}
\end{frame}

\begin{frame}
  \frametitle{Echelon Form}
  Consider the system of linear equations from the previous slides:
  \begin{equation}
    \label{eq:angeezuh}
    \left[
      \begin{array}{ccc}
        3 & 4 & -2 \\
        1 & 2 & -1 \\
        3 & -1 & -3
      \end{array}\right]\cdot\left[
      \begin{array}{c}
        x \\
        y \\
        z
      \end{array}\right]=\left[
      \begin{array}{c}
        20 \\
        9 \\
        6
      \end{array}\right]
  \end{equation}
  Combine the coefficient matrix and the vector of constants to an
  \alert{augmented matrix}:
  \begin{equation}
    \label{eq:iezaayoo}
    \left[
      \begin{array}{cccc}
        3 & 4 & -2 & 20 \\
        1 & 2 & -1 & 9 \\
        3 & -1 & -3 & 6
      \end{array}\right]
  \end{equation}
Now use elementary row operations to make sure that only zeroes
populate the matrix below the diagonal. This is called the
\alert{echelon form} of the system.
\begin{equation}
  \label{eq:eitaewoo}
    \left[
      \begin{array}{cccc}
        3 & 4 & -2 & 20 \\
        0 & -10 & 5 & -35 \\
        0 & 0 & 7 & -7
      \end{array}\right]
\end{equation}
\end{frame}

\begin{frame}
  \frametitle{Echelon Form}
  Because the echelon form is the product of elementary row
  operations, the solutions of the system of linear equations
  associated with it are the same as the solutions of the original
  system of linear equations.
  \begin{equation}
    \label{eq:thiajaxe}
    \begin{array}{ccccccc}
      3x & + & 4y & - & 2z & = & 20 \\
       &  & -10y & + & 5z & = & -35 \\
      & & & & 7z & = & -7
    \end{array}
  \end{equation}
The last equation tells us that $z=-1$. Substituting $z=-1$, the
middle equation tells us that $y=3$. Substituting both of these
results in the first equation tells us that $x=2$.
\end{frame}

\begin{frame}
  \frametitle{Echelon Form}
  The echelon form provides another way to solve a system of linear
  equations. The elementary row operations are called \alert{Gaussian
    elimination} or \alert{Gauss-Jordan elimination} (there are
  technical details about the difference between these two elimination
  methods that we are not worried about right now). Gaussian or
  Gauss-Jordan elimination, however, is hard to do manually. We use
  the echelon form primarily to deal with pathological cases where the
  determinant of the coefficient matrix is zero.
\end{frame}

\begin{frame}
  \frametitle{Theory of Linear Systems}
  \begin{enumerate}
  \item If some row of an echelon form has its first nonzero entry in
    the last column, then the system has no solution. The system is
    \alert{inconsistent}. 
  \item If a system is consistent, it has a \alert{rank}. The rank is
    the number of leading nonzero entries with respect to the rows of
    the echelon form.
  \item If the rank of a system equals the number of rows (or the
    number of equations), then the system has exactly one solution.
  \item If the rank of a system is strictly less than the number of
    rows (or the number of equations), then the system has infinitely
    many solutions.
  \end{enumerate}
\end{frame}

\begin{frame}
  \frametitle{End of Lesson}
Next Lesson: Vectors
\end{frame}

\end{document}

