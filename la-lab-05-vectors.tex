% la-lab-05-vectors.tex

\documentclass[11pt]{article}
\usepackage{enumerate}
\usepackage{syllogism} 
\usepackage{october}
\usepackage[table]{xcolor}

\newcounter{aufg}
\setcounter{aufg}{0}
\newcommand{\aufgabe}[0]{\refstepcounter{aufg}\textbf{(\arabic{aufg})}}

\begin{document}

\textbf{Vectors}

{\aufgabe} Consider the vector space of quadratic polynomials
\begin{equation}
  \label{eq:pahreeth}
V=\{f|f(x)=ax^{2}+bx+c\}  
\end{equation}
Are the following three quadratic polynomials a basis for $V$?
\begin{equation}
  \label{eq:hohnooph}
  \begin{array}{rcl}
    f_{1}(x)&=&3x^{2}+x+6 \\
    f_{2}(x)&=&5x^{2}+x+11 \\
    f_{3}(x)&=&-2x^{2}-6x+4
  \end{array}
\end{equation}
\begin{itemize}
\item If yes, find the coordinates in terms of this basis for $7x^{2}-x-4$.
\item If no, express one of the $f_{i}$ by the others, $i=1,2,3$. 
\end{itemize}

{\aufgabe} Consider the vector space of lines in $\mathbb{R}^{3}$
going through the origin.
\begin{equation}
  \label{eq:peeyohli}
W=\{L|L:ax+by+cz=0\}  
\end{equation}
Are the following three lines a basis for $W$?
\begin{equation}
  \label{eq:iozooroo}
  \begin{array}{rcl}
    L_{1}&:&2x+5y+z=0 \\
    L_{2}&:&x-y-2z=0 \\
    L_{3}&:&-3x+4z=0
  \end{array}
\end{equation}
\begin{itemize}
\item If yes, find the coordinates in terms of this basis for $L:14x+9y-12z=0$.
\item If no, express one of the $L_{i}$ by the others, $i=1,2,3$. 
\end{itemize}

{\aufgabe} There are three points whose coordinates with measurement
errors are
  \begin{equation}
    \label{eq:aukoogha}
    \begin{array}{rcl}
      I&=&(595.74,537.76) \\
       J&=&(800.92,658.44) \\
       K&=&(302.96,168.88)
    \end{array}
  \end{equation}
From station $I$, you observe an angle of $158^{\circ}49'21''$, again
with the usual measurement error. In the next lesson, we will learn
how to adjust the coordinates based on the observed angle. In the
meantime, what is the angle between $\vec{IJ}$ and $\vec{IK}$ that you
would have expected based on the coordinates? Use the inner product.

{\aufgabe} Find an orthonormal basis for the following vector plane.
Identify the normal vector to the plane.
\begin{equation}
  \label{eq:aewashic}
  -6x+5y+8z=0
\end{equation}

{\aufgabe} Consider the vector
\begin{equation}
  \label{eq:chowiute}
  \hat{b_{1}}=\left(
    \begin{array}{c}
      -5 \\
      -7 \\
      4
    \end{array}\right)
\end{equation}
Find $b_{1}$ such that $b_{1}\Vert\hat{b_{1}}$ ($b_{1}$ and
$\hat{b_{1}}$ are parallel) and $\Vert{}b_{1}\Vert=1$. Then find
vectors $b_{2},b_{3}$ so that $B=\{b_{1},b_{2},b_{3}\}$ is an
orthonormal basis of $\mathbb{R}^{3}$. Combine $b_{1},b_{2},b_{3}$ in
a matrix $M$ and find the inverse $M^{-1}$, using software. What do
you notice?

{\aufgabe} Find the equation of the tangent plane with respect to the
unit circle at $P=(0.2,0.3,\sqrt{0.87})$. Do this two ways: first use
calculus, then use linear algebra. (Ask the instructor for hints!)

{\aufgabe} Solve the following system of linear equations. Use
software.
\begin{equation}
  \label{eq:aingethu}
  \begin{array}{ccccccc}
4a & + & b  & - & c   & = & -12 \\
3a & - & 2b & + & 4c  & = & -5  \\
-a & + & 8b & - & 14c & = & -9
  \end{array}
\end{equation}
If the system is consistent and dependent, provide your answer in the
form
\begin{equation}
  \label{eq:siechaep}
  S=\{u\in\mathbb{R}^{3}\;|\;u\mbox{ corresponds to }\vec{u}=\vec{v_{0}}+s_{1}\vec{v_{1}}+{\ldots}+s_{n}\vec{v_{n}}\} %
\end{equation}
where $n$ is the dimension of the solution space and
$s_{i}\in\mathbb{R}$ for $i=1,{\ldots},n$. Note that
$(-3,2,2)^{\intercal}$ solves the system.

{\aufgabe} Solve the following system of linear equations. Use software.
\begin{equation}
  \label{eq:iexoomoh}
  \begin{array}{ccccccccc}
6x  &   &     & - & 4z & + & 5w & = & 23  \\
-3x & + & 8y  & + & 3z & - & w  & = & -14 \\
9x  & + & 8y  & - & 5z & + & 9w & = & 32  \\
    &   & 16y & + & 2z & + & 3w & = & -5
  \end{array}
\end{equation}
If the system is consistent and dependent, provide your answer in the
form
\begin{equation}
  \label{eq:xoomoube}
  S=\{u\in\mathbb{R}^{4}\;|\;u\mbox{ corresponds to }\vec{u}=\vec{v_{0}}+s_{1}\vec{v_{1}}+{\ldots}+s_{n}\vec{v_{n}}\}
\end{equation}
where $n$ is the dimension of the solution space and
$s_{i}\in\mathbb{R}$ for $i=1,{\ldots},n$. Note that
$(2,-1,1,3)^{\intercal}$ solves the system.

{\aufgabe} Find all interior angles for and the plane equation
  containing the triangle with points
  \begin{equation}
    \label{eq:yeibieba}
    P=(-6,-2,-7),Q=(-2,1,6),R=(-8,3,-5)
  \end{equation}
  Hint: Use the dot product to find the interior angles. Use the cross
  product to find a normal vector to the plane. Remember that the
  cross product of two vectors is
  \begin{equation}
    \label{eq:abeekohc}
    \vec{v}\times\vec{w}=\left\vert
      \begin{array}{ccc}
        \vec{i} & \vec{j} & \vec{k} \\
        v_{1} & v_{2} & v_{3} \\
        w_{1} & w_{2} & w_{3}
      \end{array}\right\vert
  \end{equation}
and is perpendicular to both $\vec{v}$ and $\vec{w}$. If
$P=(p_{x},p_{y},p_{z})$ is a point on a plane and
$\vec{n}=(n_{x},n_{y},n_{z})^{\intercal}$ is a normal vector to the
plane, then the plane equation is (why?)
\begin{equation}
  \label{eq:eweemeez}
  n_{x}(p_{x}-x)+n_{y}(p_{y}-y)+n_{z}(p_{z}-z)=0
\end{equation}

\end{document}

Solve the following system of linear equations. 
\begin{equation}
  \begin{array}{ccccccc}
2a & - & 6b  & - & 3c   & = & 13 \\
-5a & - & 3b & + & c  & = & 15  \\
19a & - & 3b & - & 9c & = & -19
  \end{array}\notag
\end{equation}
If the system is consistent and dependent, provide your answer in the
form
\begin{equation}
  S=\{u\in\mathbb{R}^{3}\;|\;u\mbox{ corresponds to }\vec{u}=\vec{v_{0}}+s_{1}\vec{v_{1}}+{\ldots}+s_{n}\vec{v_{n}}\}\notag
\end{equation}
where $n$ is the dimension of the solution space and
$s_{i}\in\mathbb{R}$ for $i=1,{\ldots},n$. Note that
$(-1,-3,1)^{\intercal}$ solves the system.

Solve the following system of linear equations. 
\begin{equation}
  \begin{array}{ccccccc}
4x & - & y  & + & 2z   & = & -8 \\
-2x & + & 3y & + & 7z  & = & 17  \\
8x & + & 3y & + & 20z & = & 10
  \end{array}\notag
\end{equation}
If the system is consistent and dependent, provide your answer in the
form
\begin{equation}
  S=\{u\in\mathbb{R}^{3}\;|\;u\mbox{ corresponds to }\vec{u}=\vec{v_{0}}+s_{1}\vec{v_{1}}+{\ldots}+s_{n}\vec{v_{n}}\}\notag
\end{equation}
where $n$ is the dimension of the solution space and
$s_{i}\in\mathbb{R}$ for $i=1,{\ldots},n$. Note that
$(-2,2,1)^{\intercal}$ solves the system.


Consider the vector space of parabolas with the equation\n\\$y=a(x-h)^{2}+k\\$. Are the following three parabolas a basis for this\nvector space?\n\\begin{equation}\n  \\label{eq:iozooroo}\n  \\begin{array}{ccccc}\n$v006[$i1]\n  \\end{array}\\notag\n\\end{equation}\n\\begin{itemize}\n\\item If yes, find the coordinates in terms of this basis for \\$y=$v007[$i1]\\$.\n\\item If no, express one of the three given parabolas by the other two.\n\\end{itemize}

Consider the vector space of parabolas with the equation\n\\$y=a(x-h)^{2}+k\\$. Are the following three parabolas a basis for this\nvector space?\n\\begin{equation}\n  \\label{eq:iozooroo}\n  \\begin{array}{ccccc}\n$v006[$i1]\n  \\end{array}\\notag\n\\end{equation}\n\\begin{itemize}\n\\item If yes, find the coordinates in terms of this basis for \\$y=$v007[$i1]\\$.\n\\item If no, express one of the three given parabolas by the other two.\n\\end{itemize}\n\n