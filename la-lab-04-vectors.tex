% la-lab-04-vectors.tex

\documentclass[11pt]{article}
\usepackage{enumerate}
\usepackage{syllogism} 
\usepackage{october}
\usepackage[table]{xcolor}

\newcounter{aufg}
\setcounter{aufg}{0}
\newcommand{\aufgabe}[0]{\refstepcounter{aufg}\textbf{(\arabic{aufg})}}

\begin{document}

\textbf{Complex Numbers and Vectors}

{\aufgabe} Solve the following equation for $x$ and $y$.
\begin{equation}
  \label{eq:ilishaih}
(x+y\cdot{}i)(7\cdot{}i-4)=i\cdot{}(x-5)  
\end{equation}

{\aufgabe} Convert the following expression to rectangular form with
and without Euler's formula,
\begin{equation}
  \label{eq:ookoohos}
  \frac{3+\sqrt{-4}}{4-i}
\end{equation}

{\aufgabe} Perform the indicated operations.
\begin{equation}
  \label{eq:oobaisoo}
(1-i)^{10}  
\end{equation}
\begin{equation}
  \label{eq:quiequae}
(\sqrt{3}+i)^{8}(1+i)^{5}  
\end{equation}
\begin{equation}
  \label{eq:uchaelex}
(\sqrt{3}-i)^{-8}  
\end{equation}

{\aufgabe} Show that $\frac{1}{2}(1+\sqrt{-3})$ is the reciprocal of
its conjugate. (The reciprocal of $x$ is $1/x$.)

% SOLUTION
% \begin{equation}
%   \label{eq:feachohl}
%   \frac{1}{\frac{1}{2}(1-i\sqrt{3})}=\frac{2\cdot(1+i\sqrt{3})}{(1-i\sqrt{3})(1+i\sqrt{3})}=\frac{1}{2}(1+i\sqrt{3})
% \end{equation}

{\aufgabe} Find the fifth roots of $\sqrt{-1024}$.

% SOLUTION
% \begin{equation}
%   \label{eq:queigiez}
%   \sqrt[5]{\sqrt{-1024}}=(32i)^{\frac{1}{5}}=\left(32\cdot(\cos(90^{\circ})+i\sin(90^{\circ}))\right)^{\frac{1}{5}}=\left(32\cdot{}e^{i\cdot{}90^{\circ}}\right)^{\frac{1}{5}}=2\cdot\left(\cos(18^{\circ})+i\sin(18^{\circ})\right)
% \end{equation}

% A=matrix([[n(real_part(2*(cos(18*pi/180)+i*sin(18*pi/180)))),n(imag_part(2*(cos(18*pi/180)+i*sin(18*pi/180))))],[n(real_part(2*(cos(90*pi/180)+i*sin(90*pi/180)))),n(imag_part(2*(cos(90*pi/180)+i*sin(90*pi/180))))],[n(real_part(2*(cos(162*pi/180)+i*sin(162*pi/180)))),n(imag_part(2*(cos(162*pi/180)+i*sin(162*pi/180))))],[n(real_part(2*(cos(234*pi/180)+i*sin(234*pi/180)))),n(imag_part(2*(cos(234*pi/180)+i*sin(234*pi/180))))],[n(real_part(2*(cos(306*pi/180)+i*sin(306*pi/180)))),n(imag_part(2*(cos(306*pi/180)+i*sin(306*pi/180))))]])

{\aufgabe} Consider the vector space of quadratic polynomials
\begin{equation}
  \label{eq:pahreeth}
V=\{f|f(x)=ax^{2}+bx+c\}  
\end{equation}
Are the following three quadratic polynomials a basis for $V$?
\begin{equation}
  \label{eq:hohnooph}
  \begin{array}{rcl}
    f_{1}(x)&=&3x^{2}+x+6 \\
    f_{2}(x)&=&5x^{2}+x+11 \\
    f_{3}(x)&=&-2x^{2}-6x+4
  \end{array}
\end{equation}
\begin{itemize}
\item If yes, find the coordinates in terms of this basis for $7x^{2}-x-4$.
\item If no, express one of the $f_{i}$ by the others, $i=1,2,3$. 
\end{itemize}

\newpage

{\aufgabe} Consider the vector space of lines in $\mathbb{R}^{3}$
going through the origin.
\begin{equation}
  \label{eq:peeyohli}
W=\{L|L:ax+by+cz=0\}  
\end{equation}
Are the following three lines a basis for $W$?
\begin{equation}
  \label{eq:iozooroo}
  \begin{array}{rcl}
    L_{1}&:&2x+5y+z=0 \\
    L_{2}&:&x-y-2z=0 \\
    L_{3}&:&-3x+4z=0
  \end{array}
\end{equation}
\begin{itemize}
\item If yes, find the coordinates in terms of this basis for $L:14x+9y-12z=0$.
\item If no, express one of the $L_{i}$ by the others, $i=1,2,3$. 
\end{itemize}

\end{document}

