% la-lab-03-complex.tex

\documentclass[11pt]{article}
\usepackage{enumerate}
\usepackage{syllogism} 
\usepackage{october}
\usepackage[table]{xcolor}

\newcounter{aufg}
\setcounter{aufg}{0}
\newcommand{\aufgabe}[0]{\refstepcounter{aufg}\textbf{(\arabic{aufg})}}

\begin{document}

\textbf{Echelon Form and Complex Numbers}

{\aufgabe} Find the equation of the quadratic
  \begin{equation}
    \label{eq:kieshiec}
    y=ax^{2}+bx+c
  \end{equation}
  which passes through $(0,0),(1,2),(-1,3)$.

  {\aufgabe} Find the determinant for the following Hermitian matrix.
  What do you notice?
  \begin{equation}
    \label{eq:cahkaedi}
    \left[
      \begin{array}{ccc}
        4 & 3-2i & -3i \\
        3+2i & 1 & -5+2i \\
        3i & -5-2i & 2
      \end{array}\right]
  \end{equation}

  {\aufgabe} Solve the following systems of linear equations.
  \begin{equation}
    \label{eq:yoveerae}
    \begin{array}{ccccccc}
      3x&-&4y&+&7z&=&-23 \\
      5x&-&10y&+&11z&=&-47 \\
      5x&&&-&z&=&7 \\
      x&-&3y&+&2z&=&-12
    \end{array}
  \end{equation}
  \begin{equation}
    \label{eq:jawahrei}
    \begin{array}{ccccccc}
      2u&+&v&-&2w&=&4 \\
      2u&+&4v&-&3w&=&9 \\
      4&+&5v&-&5w&=&-11
    \end{array}
  \end{equation}

  {\aufgabe} Represent the following complex numbers in polar form
  $r(\cos\theta+i\sin\theta)$.
  \begin{enumerate}
  \item $3+4i$
  \item $-1.04-1.56i$
  \end{enumerate}

  {\aufgabe} Solve the following problems in electrical engineering.
  \begin{enumerate}
  \item The impedance $Z$ (in $\Omega$) in an alternating-current
  circuit is given by$Z=3560/-32.4^{\circ}$. Express this in
  rectangular form.
  \item The current in a microprocessor circuit is represented
    by $3.75/15.0^{\circ}\mu$A. Write this in rectangular form.
  \item The voltage of a generator is represented by $2.84-1.06i$kV.
    Write this voltage in polar form.
  \end{enumerate}

  {\aufgabe} Use complex arithmetic to provide the following in
  rectangular form. 
  \begin{equation}
    \label{eq:lahqueeh}
    (8i-5)(7+4i)
  \end{equation}

  \begin{equation}
    \label{eq:usaideix}
    (\sqrt{-18}\sqrt{-4})\cdot{}3i
  \end{equation}

  \begin{equation}
    \label{eq:noemanie}
    (1+i)(1-i)^{2}
  \end{equation}

  \begin{equation}
    \label{eq:chaiyaet}
    \frac{0.25}{3-\sqrt{-1}}
  \end{equation}

  \begin{equation}
    \label{eq:guphuphe}
    \frac{6+5i}{3-4i}
  \end{equation}

  \begin{equation}
    \label{eq:aihiisha}
    \frac{(2-i^{3})^{4}}{(i^{8}-i^{6})^{3}}+i
  \end{equation}
    
\end{document}

