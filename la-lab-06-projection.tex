% la-lab-06-projection.tex

\documentclass[11pt]{article}
\usepackage{enumerate}
\usepackage{syllogism} 
\usepackage{october}
\usepackage[table]{xcolor}

\newcounter{aufg}
\setcounter{aufg}{0}
\newcommand{\aufgabe}[0]{\refstepcounter{aufg}\textbf{(\arabic{aufg})}}

\begin{document}

\textbf{Projection}

{\aufgabe} What is the projection of
  $\vec{u}=(5,2,10)^{\intercal}$ onto the plane $H$ characterized by
  $2x+y+3z=0$?

  Procedure:

  \begin{enumerate}
  \item Find a vector in the plane. Choose $x,y$ arbitrarily, then
    calculate $z$. Call this vector $v_{1}$. 
  \item Find another vector in the plane and make sure it is linearly
    independent of $v_{1}$. Call it $v_{2}$. Now you have a basis for $H$.
  \item You know the following about $u_{H}$. Use it to find the
    coordinates of $u_{H}$ by forming a system of linear equations.
    \begin{enumerate}
    \item $u_{H}$ is in the plane
    \item $u-u_{H}\perp{}v_{1}$
    \item $u-u_{H}\perp{}v_{2}$
    \end{enumerate}
  \end{enumerate}

{\aufgabe} Let $u$ and $v$ be some linearly independent vectors. Then
the formula for $u_{H}$, where $H$ is the hyperplane spanned by the
basis $\{v\}$, is
\begin{equation}
  \label{eq:aechachu}
    u_{H}=\left(\frac{u\cdot{}v}{v\cdot{}v}\right)v
\end{equation}
This only works for one-dimensional $H$! Show that it is true by
writing $u_{H}=av$ for some $a\in\mathbb{R}$ and isolating $a$ in
\begin{equation}
  \label{eq:oatiefai}
  (u-av)\perp{}v\notag
\end{equation}

{\aufgabe} Let $H=\mbox{span}(\{v\})$ with $v=(-2,3)^{\intercal}$.
Find $u_{H}$ for $u=(7,5)^{\intercal}$.

{\aufgabe} Let's try \textbf{(1)} again with a different strategy:
What is the projection of $\vec{u}=(5,2,10)^{\intercal}$ onto the
plane $H$ characterized by $2x+y+3z=0$?

  Procedure:

  \begin{enumerate}
  \item Find an orthogonal basis for $H$. (In the above procedure,
    find $v_{2}$ such that $v_{2}\perp{}v_{1}$.)
  \item Note that the following is true (but only for a basis where
    the basis vectors $v_{i}$ are pairwise orthogonal!):
    $u_{H}=u_{v_{1}}+{\ldots}+u_{v_{n}}$. Show that it is true for
    $n=2$, i.e.\
      \begin{enumerate}
  \item $(u_{v_{1}}+u_{v_{2}})\in{}H$ (trivial)
  \item $(u-(u_{v_{1}}+u_{v_{2}}))\perp{}v_{1}$ (use the fact that $v_{1}\perp{}v_{2}$)
  \item $(u-(u_{v_{1}}+u_{v_{2}}))\perp{}v_{2}$ (same idea)
  \end{enumerate}
  \item Use formula (\ref{eq:aechachu}) to calculate $u_{v_{1}}$ and
    $u_{v_{2}}$. 
  \item Use $u_{v_{1}}+u_{v_{2}}=u_{H}$ to find $u_{H}$.
  \end{enumerate}

  {\aufgabe} Consider the system of non-linear equations
  \begin{equation}
    \label{eq:ukaephei}
    \begin{array}{ccccc}
      3^{x}&-&y^{2}&=&2 \\
      xy&+&\cos(y-5)&=&16
    \end{array}
  \end{equation}
To solve it numerically, we need to linearize the following function
around an initial estimate of the solution $x=4,y=4$.
\begin{equation}
  \label{eq:aeluquoh}
  f((x,y)^{\intercal})=\left(
    \begin{array}{c}
      e^{x\ln{}3}-y^{2}-2 \\
      xy+\cos(y-5)-16
    \end{array}\right)
\end{equation}
Find the Jacobian of $f$ and derive the linearization of the function
around $x=4,y=4$.

\end{document}
