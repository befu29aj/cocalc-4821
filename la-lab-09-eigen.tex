% la-lab-09-eigen.tex

\documentclass[11pt]{article}
\usepackage{enumerate}
\usepackage{syllogism} 
\usepackage{october}
\usepackage[table]{xcolor}

\newcounter{aufg}
\setcounter{aufg}{0}
\newcommand{\aufgabe}[0]{\refstepcounter{aufg}\textbf{(\arabic{aufg})}}

\begin{document}

\textbf{Eigenvalues and Normal Distribution}

{\aufgabe} Find an invertible matrix $P$ such that $P^{-1}AP$ is
diagonal.
\begin{equation}
  \label{eq:vaitaino}
  A=\left[
    \begin{array}{cc}
      -1 & 0 \\
      2 & 4
    \end{array}\right]\notag
\end{equation}

{\aufgabe} Find an invertible matrix $P$ such that $P^{-1}AP$ is
diagonal.
\begin{equation}
  \label{eq:miezahxu}
  A=\left[
    \begin{array}{cc}
      2 & 1 \\
      -6 & 7
    \end{array}\right]\notag
\end{equation}

{\aufgabe} Find an invertible matrix $P$ such that $P^{-1}AP$ is
diagonal.
\begin{equation}
  \label{eq:lahchahd}
  A=\left[
    \begin{array}{ccc}
      4 & 0 & 0 \\
      3 & 2 & 0 \\
      0 & 2 & 1
    \end{array}\right]\notag
\end{equation}

{\aufgabe} Find the eigenvalues of 
\begin{equation}
  \label{eq:kahnuame}
  B=\left[
    \begin{array}{ccc}
      -9 & -11 & -17 \\
      16 & 18 & 25 \\
      -4 & -4 & -5
    \end{array}\right]\notag
\end{equation}
Use the formula for the characteristic polynomial of a 3x3 matrix
\begin{equation}
  \label{eq:oofohnga}
p(\lambda)=-\lambda^{3}+\mbox{tr}(A)\lambda^{2}+\left(\mbox{tr}(A)^{2}-\mbox{tr}(A^{2})\right)\lambda+\det(A)\notag
\end{equation}
Recall that $\mbox{tr}(A)$ is the trace of $A$, the sum of the
diagonal elements of $A$. To solve the cubic equation, try to guess a
whole number solution and use polynomial division. Alternatively, use
a graphing utility.

{\aufgabe} Find $A^{5}$ for
  \begin{equation}
    \label{eq:aisejiez}
    A=\left[
      \begin{array}{cc}
        19&-12 \\
        24&-15
      \end{array}\right]\notag
  \end{equation}
using similar matrices.

{\aufgabe} Find the area under the curve for the following sets of
$z$-scores. 
\begin{equation}
  \label{eq:deapheph}
\{z|z\leq{}-1.72\}  \notag
\end{equation}
\begin{equation}
  \label{eq:taedaiga}
\{z|1.96<{}z\}  \notag
\end{equation}
\begin{equation}
  \label{eq:ahraefis}
\{z|-1.55\leq{}z\leq{}-0.81\}\notag
\end{equation}

{\aufgabe} Find the area under the curve for the
following sets of $x$-values. 
\begin{equation}
  \label{eq:aequaixe}
\{x|x\leq{}83\},\mu=100,\sigma=15\notag
\end{equation}
\begin{equation}
  \label{eq:oocohdau}
\{x|0.44<x\},\mu=0.5,\sigma=0.2  \notag
\end{equation}
\begin{equation}
  \label{eq:aethohph}
\{x|800\leq{}x\leq{}1200\},\mu=911,\sigma=121\notag
\end{equation}

{\aufgabe} According to 2017 census data, the United States have a
population of 325,719,178. Calculate how many of these people you
would expect to have an IQ that is higher than 145
($\mu=100,\sigma=15$). 

{\aufgabe} My bicycle commute from home to BCIT is normally
distributed in terms of the time it takes to complete. The mean is
$\mu=27.3$ minutes, the standard deviation is $\sigma=2.16$ minutes.
If I leave my house at 7:50am and I have to be at BCIT at 8:20am in
order to be on time for my class, what is the probability that I will
be late for class?

\end{document}
