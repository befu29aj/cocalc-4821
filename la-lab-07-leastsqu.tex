% la-lab-07-leastsqu.tex

\documentclass[11pt]{article}
\usepackage{enumerate}
\usepackage{syllogism} 
\usepackage{october}
\usepackage[table]{xcolor}

\newcounter{aufg}
\setcounter{aufg}{0}
\newcommand{\aufgabe}[0]{\refstepcounter{aufg}\textbf{(\arabic{aufg})}}

\begin{document}

\textbf{Least Squares}

{\aufgabe} Consider the following function:
\begin{equation}
f\left(\left[
\begin{array}{c}
x \\
y
\end{array}\right]\right)=\left[
\begin{array}{c}
x^{2}+x\sin(x+y) \\
\sin{}x\cos(x+y)
\end{array}\right]
\end{equation}
Linearize the function around $x=\frac{\pi}{2},y=\frac{\pi}{2}$ so it looks as follows,
\begin{equation}
f(x)\approx{}E+\left[
\begin{array}{cc}
A&B \\
C&D
\end{array}\right]\left[
\begin{array}{c}
x-M \\
x-N
\end{array}\right]
\end{equation}
Specify the numbers $A,B,C,D,E,M,N$ in your solution.

{\aufgabe} Project the vector $y=(4,-6,-10,-10)^{\intercal}$ onto the
plane containing the points
\begin{equation}
  \label{eq:eareisoo}
  P=(-7,8,3,0),Q=(-3,11,2,6),R=(-4,11,12,-2)
\end{equation}

{\aufgabe} Project the vector $y=(4,-6,-10,-10)^{\intercal}$ onto the
plane containing the points
\begin{equation}
  \label{eq:eareisoo}
  P=(-7,8,3,0),Q=(-3,11,2,6),R=(-2,11,-8,14)
\end{equation}
You should get the same answer as in the last question, but your
procedure will probably be different.

{\aufgabe} At 6327 ft (or 6.327 thousand feet), Mario Triola recorded
  the temperature. Find the best predicted temperature at that
  altitude based on other measurements, assuming a linear
  relationship. How does the result compare to the actual recorded
  value of 48$^{\circ}$F?

\begin{tabular}{|l|l|l|l|l|l|l|l|}
\hline
Altitude & 3 & 10 & 14 & 22 & 28 & 31 & 33 \\
\hline
Temperature & 57 & 37 & 24 & -5 & -30 & -41 & -54 \\
\hline
\end{tabular}

Do this both ways, using the $Y-AV=E$ setup and projection on the one
hand and, on the other hand, the formula
\begin{equation}
  \label{eq:iduochuk}
    V_{0}=\left[
      \begin{array}{c}
        m \\
        b
      \end{array}\right]=(A^{\intercal}A)^{-1}A^{\intercal}Y  
\end{equation}

{\aufgabe} You measure the four angles in a quadrilateral
$\hat{a}=8.490426^{\circ},\hat{b}=182.029154^{\circ},\hat{c}=119.148088^{\circ},\hat{d}=50.32948$.
What are the least squares adjusted measurements?

{\aufgabe} Consider the following leveling network. 
    \begin{figure}[h]
    \includegraphics[scale=0.26]{./diagrams/levelingnetwork.png}
  \end{figure}
  The objective is to determine elevations of $A$, $B$, and $C$, which
  are to serve as temporary project bench marks to control
  construction of a highway through the crosshatched corridor.

  Obviously, it would have been possible to obtain elevations for $A$,
  $B$, and $C$ by beginning at $BMX$ and running a single closed loop
  consisting of only courses 1, 5, 7, and 4. Alternatively, a single
  closed loop could have been initiated at $BMY$ and consist of courses
  2, 5, 7, and 3.

  However, by running all seven courses, redundancy is achieved that
  enables checks to be made, blunders to be isolated, and precision to
  be increased. Having run all seven courses, it would be possible to
  compute the adjusted elevation of $B$, for example, using several
  different single closed circuits. Loops 1-5-6, 2-5-6, 3-7-6, and
  4-7-6 could each be used, but it is almost certain that each would
  yield a different elevation for $B$.

  A more logical approach, which will produce only one adjusted value
  for $B$---its most probable one---is to use all seven courses in a
  simultaneous least squares adjustment. In adjusting level networks,
  the observed difference in elevation for each course is treated as
  one observation containing a single random error.

  This single random error is the total of the individual random
  errors in backsight and foresight readings for the entire course. In
  the figure, the arrows indicate the direction of leveling. Thus, for
  course number 1, leveling proceeded from $BMX$ to $A$ and the observed
  elevation difference was $+5.10$ feet.

Calculate the least squares
  adjusted elevations of $A$, $B$, and $C$ using the following
  observation equations.
  \begin{equation}
    \label{eq:aelaeghu}
    \begin{array}{rcl}
      A&=&BMX+5.10+\epsilon_{1} \\
      BMY&=&A+2.34+\epsilon_{2} \\
      C&=&BMY-1.25+\epsilon_{3} \\
      BMX&=&C-6.13+\epsilon_{4} \\
      B&=&A-0.68+\epsilon_{5} \\
      B&=&BMY-3.00+\epsilon_{6} \\
      C&=&B+1.70+\epsilon_{7}
    \end{array}
  \end{equation}

\end{document}
