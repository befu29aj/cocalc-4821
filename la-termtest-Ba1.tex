% la-termtest-Ba1.tex

\documentclass[11pt]{article}
\usepackage{alltt}
\usepackage{enumerate}
\usepackage{syllogism} 
\usepackage{october}
\usepackage[table]{xcolor}
\pagestyle{empty}

\newcounter{aufg}
\setcounter{aufg}{0}
\newcommand{\aufgabe}[1]{\refstepcounter{aufg}\textbf{(\arabic{aufg})} [#1 points]}

\begin{document}

\textbf{Term Test Ba version 1}

\aufgabe{5} Consider the vector space of 2x2 matrices. Are the following three matrices a basis for this vector space?
\begin{equation}
A=\left[
\begin{array}{cc}
-9 & -4 \\
-3 & 2
\end{array}\right],B=\left[
\begin{array}{cc}
-5 & 2 \\
-6 & -5
\end{array}\right],C=\left[
\begin{array}{cc}
1 & -8 \\
9 & 12
\end{array}\right],D=\left[
\begin{array}{cc}
4 & -8 \\
 -2 & 7
\end{array}\right]\notag
\end{equation}
\begin{itemize}
\item If yes, find the coordinates in terms of this basis for 
\begin{equation}
D=\left[
\begin{array}{cc}
-9 & 2 \\
1 & 7
\end{array}\right]\notag
\end{equation}

\item If no, express one of the four given matrices by the other three.
\end{itemize}

\aufgabe{5} Solve the following system of linear equations.
\begin{equation}
\begin{array}{ccccccc}
2a&-&6b&-&3c&=&13\\
-5a&-&3b&+&c&=&15\\
19a&-&3b&-&9c&=&-19
\end{array}\notag
\end{equation}
If the system is consistent and dependent, provide your answer in the form
\begin{equation}
S=\{u\in\mathbb{R}^{3}\;|\;u\mbox{ corresponds to }\vec{u}=\vec{v_{0}}+s_{1}\vec{v_{1}}+{\ldots}+s_{n}\vec{v_{n}}\}\notag
\end{equation}
where $n$ is the dimension of the solution space and $s_{i}\in\mathbb{R}$ for $i=1,{\ldots},n$. Note that
$(-1,-3,1)^{\intercal}$ solves the system.

\aufgabe{5} Consider the following three vectors in $\mathbb{R}^{3}$,
\begin{equation}
\left(
\begin{array}{c}
-7\\
-2\\
3

\end{array}\right),\left(
\begin{array}{c}
-6\\
-10\\
-2

\end{array}\right),\left(
\begin{array}{c}
10\\
-3\\
7

\end{array}\right)\notag
\end{equation}
 Determine the three lengths of these vectors and the three angles between them. If they replace the origin to the points $P,Q,R$, determine the plane equation for the plane containing the three points, using the cross product.

\end{document}
