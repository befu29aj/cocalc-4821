% la-01-labasics.tex

\documentclass[xcolor=dvipsnames]{beamer}
\usepackage{teachbeamer}

\title{Matrix Basics}
\subtitle{{\CourseNumber}, BCIT}

\author{\CourseName}

\date{September 10, 2018}

\begin{document}

\begin{frame}
  \titlepage
\end{frame}

\begin{frame}
  \frametitle{Matrix Definition}
  A matrix is a tabular arrangement of real numbers. 
  \begin{equation}
    \label{eq:oathaemo}
    A=\left[\begin{array}{cccc}
    a_{11}&a_{12}&\cdots{}&a_{1n} \\
          a_{21}&\ddots{}&& \\
          \vdots{}&&&\vdots \\
          a_{m1}&&\cdots{}&a_{mn}
  \end{array}\right]
\end{equation}
The number of rows is $m$, the number of columns is $n$. 
\end{frame}

\begin{frame}
  \frametitle{Matrix Addition}
We can define operations on matrices just like we define operations on
numbers. For example, we can add an $m\times{}n$ matrix to another one
as follows,
\begin{equation}
  \label{eq:ahloongi}
  \left[\begin{array}{cccc}
    a_{11}&a_{12}&\cdots{}&a_{1n} \\
          a_{21}&\ddots{}&& \\
          \vdots{}&&&\vdots \\
          a_{m1}&&\cdots{}&a_{mn}
  \end{array}\right]+\left[
\begin{array}{cccc}
    b_{11}&b_{12}&\cdots{}&b_{1n} \\
          b_{21}&\ddots{}&& \\
          \vdots{}&&&\vdots \\
          b_{m1}&&\cdots{}&b_{mn}
  \end{array}\right]=\notag
\end{equation}
\begin{equation}
  \label{eq:fohghoaw}
\left[
\begin{array}{cccc}
    a_{11}+b_{11}&a_{12}+b_{12}&\cdots{}&a_{1n}+b_{1n} \\
          a_{21}+b_{21}&\ddots{}&& \\
          \vdots{}&&&\vdots \\
          a_{m1}+b_{m1}&&\cdots{}&a_{mn}+b_{mn}
  \end{array}\right]\notag
\end{equation}
\end{frame}

\begin{frame}
  \frametitle{Matrix Addition}
  \beispiel{Adding and Subtracting Matrices}
  \begin{equation}
    \label{eq:pahleuza}
    \left[
      \begin{array}{cc}
        1 & -2 \\
        -3 & -1 
      \end{array}\right]+\left[
      \begin{array}{cc}
        3 & -3 \\
        4 & 1
      \end{array}\right]=\left[
      \begin{array}{cc}
        4 & -5 \\
        1 & 0
      \end{array}\right]\notag
  \end{equation}
  \begin{equation}
    \label{eq:pahleuza}
    \left[
      \begin{array}{cc}
        5 & -6 \\
        -2 & 2 
      \end{array}\right]-\left[
      \begin{array}{cc}
        -6 & 5 \\
        0 & 7
      \end{array}\right]=\left[
      \begin{array}{cc}
        11 & -11 \\
        -2 & -5
      \end{array}\right]\notag
  \end{equation}
\end{frame}

\begin{frame}
  \frametitle{Matrix Scalar Multiplication}
Next, we define what it means to multiply a matrix by a scalar, i.e.\
a real number (NOT a matrix). 
\begin{equation}
  \label{eq:theishie}
  k\cdot\left[\begin{array}{cccc}
    a_{11}&a_{12}&\cdots{}&a_{1n} \\
          a_{21}&\ddots{}&& \\
          \vdots{}&&&\vdots \\
          a_{m1}&&\cdots{}&a_{mn}
  \end{array}\right]=\left[\begin{array}{cccc}
    ka_{11}&ka_{12}&\cdots{}&ka_{1n} \\
          ka_{21}&\ddots{}&& \\
          \vdots{}&&&\vdots \\
          ka_{m1}&&\cdots{}&ka_{mn}
  \end{array}\right]\notag
\end{equation}
\end{frame}

\begin{frame}
  \frametitle{Matrix Scalar Multiplication}
  \beispiel{Multiplying a Matrix by a Scalar}
  \begin{equation}
    \label{eq:pahleuza}
    2\cdot\left[
      \begin{array}{cc}
        -5 & -3 \\
        -7 & 8 
      \end{array}\right]=\left[
      \begin{array}{cc}
        -10 & -6 \\
        -14 & 16
      \end{array}\right]\notag
  \end{equation}
  \begin{equation}
    \label{eq:pahleuza}
    -\frac{1}{3}\cdot\left[
      \begin{array}{cc}
        -1 & -3 \\
        -7 & 1 
      \end{array}\right]=\left[
      \begin{array}{cc}
        \frac{1}{3} & 1 \\
        \frac{7}{3} & -\frac{1}{3}
      \end{array}\right]\notag
  \end{equation}
\end{frame}

\begin{frame}
  \frametitle{Matrix Product}
Finally, we define matrix multiplication. You can multiply an
$m\times{}j$ matrix by a $j\times{}n$ matrix, which will give you an
$m\times{}n$ matrix.
\begin{equation}
  \label{eq:orahpahn}
  \left[\begin{array}{cccc}
    a_{11}&a_{12}&\cdots{}&a_{1j} \\
          a_{21}&\ddots{}&& \\
          \vdots{}&&&\vdots \\
          a_{m1}&&\cdots{}&a_{mj}
  \end{array}\right]\cdot
\left[\begin{array}{cccc}
    b_{11}&b_{12}&\cdots{}&b_{1n} \\
          b_{21}&\ddots{}&& \\
          \vdots{}&&&\vdots \\
          b_{j1}&&\cdots{}&b_{jn}
  \end{array}\right]=\notag
\end{equation}
\begin{equation}
  \label{eq:raipuboi}
  \left[\begin{array}{cccc}
    c_{11}&c_{12}&\cdots{}&c_{1n} \\
          c_{21}&\ddots{}&& \\
          \vdots{}&&&\vdots \\
          c_{m1}&&\cdots{}&c_{mn}
  \end{array}\right]\notag
\end{equation}
where $c_{ik}=a_{i1}b_{1k}+a_{i2}b_{2k}+\ldots+a_{ij}b_{jk}$.
\end{frame}

\begin{frame}
  \frametitle{Matrix Method}
  \begin{equation}
    \label{eq:ohghohfi}
    \begin{array}{rcrcl}
      5x&+&3y&=&13.5 \\
      x&+&5y&=&13.7
    \end{array}
  \end{equation}
is the system of linear equations that we are trying to solve. A
matrix is a rectangular arrangement of numbers, for example
\begin{equation}
  \label{eq:cegeemoi}
  \left[\begin{array}{ccc}
    5&3&13.5 \\
    1&5&13.7
  \end{array}\right]
\end{equation}
There are many fascinating things you can do with matrices. The
discipline that deals with matrices is called Linear Algebra. 
\end{frame}

\begin{frame}
  \frametitle{Matrix Inverse I}
  Matrix multiplication for an an $m\times{}j$ matrix by a
  $k\times{}n$ matrix is not defined when $j\neq{}k$. An inverse
  matrix $A^{-1}$ of a square matrix $A$ is defined to be the matrix
\begin{equation}
  \label{eq:vaishien}
A\cdot{}A^{-1}=A^{-1}\cdot{}A=E
\end{equation}
where
\begin{equation}
  \label{eq:phoaxoze}
  E=\left[\begin{array}{ccccc}
     1        & 0 & \cdots{} &   & 0      \\
     0        & 1 & \cdots{} &   & 0      \\
     \vdots{} &   & \ddots{} &   & \vdots \\
     0        &   & \cdots{} & 1 & 0      \\
     0        &   & \cdots{} & 0 & 1
  \end{array}\right]\notag
\end{equation}
\end{frame}

\begin{frame}
  \frametitle{Matrices and Systems of Linear Equations I}
Remember our system of linear equations. 
  \begin{equation}
    \label{eq:ahgohcoh}
    \begin{array}{rcrcl}
      5x & + & 3y & = & 13.5 \\
      x  & + & 5y & = & 13.7
    \end{array}
  \end{equation}
In matrix notation, we can write
  \begin{equation}
    \label{eq:neithohn}
  \left[\begin{array}{cc}
5        & 3                 \\
 1       & 5
  \end{array}\right]\cdot
  \left[\begin{array}{c}
 x                 \\
 y
  \end{array}\right]=
  \left[\begin{array}{c}
13.5                 \\
13.7
  \end{array}\right]\notag
  \end{equation}
\end{frame}

\begin{frame}
  \frametitle{Matrices and Systems of Linear Equations II}
Let's call these three matrices $A,v,b$ respectively. $A$ and $b$ are
provided, and we are looking for $v$. If we had $A^{-1}$, we could go
from
\begin{equation}
  \label{eq:baixieda}
  Av=b
\end{equation}
to
\begin{equation}
  \label{eq:maethung}
  A^{-1}Av=A^{-1}b
\end{equation}
which is the same as
\begin{equation}
  \label{eq:leighuga}
  v=A^{-1}b
\end{equation}
The challenge is therefore to find $A^{-1}$. Scientific calculators
and computers can find $A^{-1}$ for you. 
\end{frame}

\begin{frame}
  \frametitle{Matrix Inverse and Determinants}
  If you want to know how to find the inverse yourself, one method to
  use is calculating the determinant of a matrix. It takes a bit of
  time to understand determinants, and then it's still a complicated
  (and not very transparent) procedure to get to the inverse. For
  $2\times{}2$ matrices, however, the inverse is
  \begin{equation}
    \label{eq:iephaizu}
    A^{-1}=\frac{1}{\det{}A}\left[
      \begin{array}{cc}
        d & -b \\
        -c & a
      \end{array}\right]
  \end{equation}
for
  \begin{equation}
    \label{eq:sooxaexa}
    A=\left[
      \begin{array}{cc}
        a & b \\
        c & d
      \end{array}\right]
  \end{equation}
and the determinant is $\det{}A=ad-bc$.
\end{frame}

\begin{frame}
  \frametitle{Matrix Row Operations}
  Another method to find the inverse of a matrix is using
  \alert{matrix row operations}. There are three matrix row
  operations.
\begin{itemize}
\item \alert{Row Switching} means you are allowed to switch two rows,
  for example $R_{1}\leftrightarrow{}R_{2}$
\item \alert{Row Multiplication} means you are allowed to multiply all
  elements of a row by a real non-zero number, for example
  $\frac{2}{5}R_{2}\rightarrow{}R_{2}$
\item \alert{Row Addition} means you are allowed to add one row to
  another and then replace one of the original rows by the sum of the
  two rows, for example $R_{1}+R_{2}\rightarrow{}R_{1}$
\end{itemize}
Row multiplication and row addition are often used together, for
example $\frac{7}{8}R_{1}-R_{3}\rightarrow{}R_{3}$.
\end{frame}

\begin{frame}
  \frametitle{Matrix Row Operations}
To find the inverse of a square matrix, we combine $A$ and $E$
  \begin{equation}
    \label{eq:aurohbac}
  \left[\begin{array}{cccc}
 5 & 3 & 1 & 0 \\
 1 & 5 & 0 & 1
  \end{array}\right]\notag
  \end{equation}
and apply matrix row operations until we get
  \begin{equation}
    \label{eq:nahshooh}
  \left[\begin{array}{cccc}
 1 & 0 & x & y \\
 0 & 1 & z & w
  \end{array}\right]\notag
  \end{equation}
where
  \begin{equation}
    \label{eq:aecocaeh}
  A^{-1}=\left[\begin{array}{cc}
 x & y  \\
 z & w 
  \end{array}\right]\notag
  \end{equation}
\end{frame}

\begin{frame}
  \frametitle{Inverse Example}
For our example,
  \begin{equation}
    \label{eq:weeraesh}
  \left[\begin{array}{cccc}
 5    & 3     & 1     & 0     \\
 1    & 5     & 0     & 1
  \end{array}\right]\longrightarrow
  \left[\begin{array}{cccc}
 25/3 & 5     & 5/3   & 0     \\
 1    & 5     & 0     & 1
        \end{array}\right]\longrightarrow\notag
\end{equation}
  \begin{equation}
  \left[\begin{array}{cccc}
 22/3 & 0     & 5/3   & -1    \\
 1    & 5     & 0     & 1
  \end{array}\right]\longrightarrow\notag
\end{equation}
  \begin{equation}
    \label{eq:ieyoongu}
  \left[\begin{array}{cccc}
 22/3 & 0     & 5/3   & -1    \\
 22/3 & 110/3 & 0     & 22/3
  \end{array}\right]\longrightarrow
  \left[\begin{array}{cccc}
 22/3 & 0     & 5/3   & -1    \\
 0    & 110/3 & -5/3  & 25/3
  \end{array}\right]\longrightarrow\notag
\end{equation}
  \begin{equation}
    \label{eq:ephoopha}
  \left[\begin{array}{cccc}
 1    & 0     & 5/22  & -3/22 \\
 0    & 1     & -1/22 & 5/22
  \end{array}\right]\notag
  \end{equation}
\end{frame}

\begin{frame}
  \frametitle{Inverse Example}
  For step 1, we multiplied the first row by $5/3$ (row
  multiplication). For step 2, we subtracted the second row from the
  first row and replaced the first row by the result (row addition).
  For step 3, we multiplied the second row by $22/3$ (row
  multiplication). For step 4, we subtracted the first row from the
  second row and replaced the second row by the result (row addition).
  For the last step, we multiplied the first row by $3/22$ and the
  second row by $3/110$ (row multiplication applied twice).
\end{frame}

\begin{frame}
  \frametitle{Matrices and Systems of Linear Equations III}
Thus,
\begin{equation}
  \label{eq:oogeujie}
  A^{-1}=\left[\begin{array}{cc}
 5/22  & -3/22 \\
 -1/22 & 5/22
               \end{array}\right]=\frac{1}{22}\cdot\left[
               \begin{array}{cc}
                 5 & -3 \\
                 -1 & 5
               \end{array}\right]\notag
\end{equation}
and
\begin{equation}
  \label{eq:seeleeje}
  v=A^{-1}b=\left[\begin{array}{cc}
 5/22  & -3/22 \\
 -1/22 & 5/22
  \end{array}\right]\cdot
\left[\begin{array}{c}
 13.5   \\
 13.7  
  \end{array}\right]=\left[\begin{array}{c}
 1.2   \\
 2.5  
  \end{array}\right]\notag
\end{equation}
\end{frame}

\begin{frame}
  \frametitle{End of Lesson}
Next Lesson: Determinants and Inverse
\end{frame}

\end{document}

