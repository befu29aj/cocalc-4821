% la-lab-08-nonlinear.tex

\documentclass[11pt]{article}
\usepackage{enumerate}
\usepackage{syllogism} 
\usepackage{october}
\usepackage[table]{xcolor}

\newcounter{aufg}
\setcounter{aufg}{0}
\newcommand{\aufgabe}[0]{\refstepcounter{aufg}\textbf{(\arabic{aufg})}}

\begin{document}

\textbf{Least Squares Adjustments with Non-Linear Equations}

  {\aufgabe} Find a matrix $C$ such that
  \begin{equation}
    \label{eq:eizuchan}
    C^{2}=A\mbox{ where }A=\left[
      \begin{array}{cc}
        3&-1 \\
        2&0
      \end{array}\right]\notag
  \end{equation}
  We will learn how to solve this problem using eigenvalues. For now,
  we are faced with a system of non-linear equations given
\begin{equation}
  \label{eq:xohdaihu}
  C=\left[
    \begin{array}{cc}
      x_{1}&x_{2} \\
       x_{3}&x_{4}
    \end{array}\right]\notag
\end{equation}
and
\begin{equation}
    \label{eq:ahxeemei}
    \begin{array}{ccccc}
x_{1}^{2}         & + & x_{2}x_{3} & = & 3    \\
x_{1}x_{2}        & + & x_{2}x_{4} & = & -1   \\
x_{1}x_{3}        & + & x_{3}x_{4} & = & 2    \\
x_{2}x_{3}        & + & x_{4}^{2}  & = & 0
    \end{array}\notag
  \end{equation}

  {\aufgabe} You are trying to measure the coordinates of stations $A$
  and $B$. Your provisional estimate is $(8.3995,3.0161)$ and
  $(-2.872,1.4937)$. Then you observe the length between $A$ and $B$
  to be $11.391$. How would you report your least squares adjusted
  coordinates for $A$ and $B$, given that you weigh equally the errors
  for $A$ and $B$'s coordinates as well as the distance between them?

    Setting up the first four yave equations is simple. The fifth one,
  however, is non-linear.
  \begin{equation}
    \label{eq:pheengei}
    \begin{array}{rcl}
      x&=&8.3995+\epsilon_{1} \\
      y&=&3.0161+\epsilon_{2} \\
      z&=&-2.872+\epsilon_{3} \\
      w&=&1.4937+\epsilon_{4} \\
      \sqrt{(z-x)^{2}+(w-y)^{2}}&=&11.391+\epsilon_{5} \\
    \end{array}\notag
  \end{equation}
  Linearize the fifth equation using the Taylor polynomial expansion
  of the function $G(x,y,z,w)=\sqrt{(z-x)^{2}+(w-y)^{2}}$.

{\aufgabe} Solve the following system of nonlinear equations
numerically. Use $(x_{0},y_{0})=(0.6,0.8)$ as your first approximation.
  \begin{equation}
    \label{eq:gieghaik}
    \begin{array}{ccccc}
      \cos{}x&-&y&=&0 \\
             x&-&y^{2}&=&0
    \end{array}\notag
  \end{equation}
The solution set is
\[S=\{(x,y)\in\mathbb{R}^{2}|x\approx{}0.64171,y\approx{}0.80107\}\]

{\aufgabe} There are three points whose
  coordinates with measurement errors are
  \begin{equation}
    \label{eq:aukoogha}
    \begin{array}{rcl}
      I&=&(595.74,537.76) \\
       J&=&(800.92,658.44) \\
       K&=&(302.96,168.88)
    \end{array}
  \end{equation}
  From station $I$, you observe an angle of $158^{\circ}49'21''$
  instead of the expected $158^{\circ}54'5.9107''$ between $\vec{IJ}$
  and $\vec{IK}$. How should you least squares adjust the coordinates
  of $I,J,K$ in light of your angle measurement? (Note that it is
  unnatural to give equal weight to the errors in coordinate
  measurements and angle measurements: this can be addressed by weight
  factors, but let us skip this step here for simplicity.)

  {\aufgabe} For the following matrices, find the eigenvalues and one
  corresponding eigenvector for each eigenvalue.
  \begin{equation}
    \label{eq:uuvohien}
    A_{1}=\left[
      \begin{array}{cc}
        22&20 \\
        -25&-23
      \end{array}\right]\hspace{.5in}A_{2}=\left[
      \begin{array}{cc}
       1&1 \\
        1&1
      \end{array}\right]\hspace{.5in}A_{3}=\left[
      \begin{array}{cc}
        1&3 \\
        1&-1
      \end{array}\right]\notag
  \end{equation}
  \begin{equation}
    \label{eq:eitujush}
    A_{4}=\left[
      \begin{array}{ccc}
        1&2&1 \\
        0&2&1 \\
        0&0&3
      \end{array}\right]\hspace{.5in}A_{5}=\left[
      \begin{array}{ccc}
        5&2&1 \\
        0&5&0 \\
        0&0&3
      \end{array}\right]\hspace{.5in}A_{6}=\left[
      \begin{array}{cccc}
        2&0&0&0 \\
        3&1&0&0 \\
        4&0&2&0 \\
        -1&1&3&6
      \end{array}\right]\notag
  \end{equation}

\end{document}
