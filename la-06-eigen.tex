% la-06-eigen.tex

\documentclass[xcolor=dvipsnames]{beamer}
\usepackage{teachbeamer}

\title{Eigenvalues and Eigenvectors}
\subtitle{{\CourseNumber}, BCIT}

\author{\CourseName}

\date{October 29, 2018}

\begin{document}

\begin{frame}
  \titlepage
\end{frame}

\begin{frame}
  \frametitle{Motivation}
Here is a list of questions that can be answered using eigenvalues and
eigenvectors.
\begin{itemize}
\item Let the probability of rain tomorrow depend only on whether
  there is rain today. If it rains today, the probability of rain
  tomorrow is 20\%. If it is clear today, the probability of rain
  tomorrow is 10\%. What is the average ratio of rainy days to clear
  days in this climate?
\item Let a particle go on a random walk along a line between $S_{1}$
  and $S_{n}$. How much of its time does it spend at $S_{i}$?
\item The Fibonacci sequence is $1,1,2,3,5,8,13,21,34,{\ldots}$. It is
  used in many applications, for example population modeling. Is there
  an explicit (not recursive) formula for the $n$-th term?
\item Given a matrix $A$, what is $A^{n}$ for large $n$?
\item Given a matrix $B$, what is a matrix $C$ such that $C^{2}=B$?
\end{itemize}
\end{frame}

\begin{frame}
  \frametitle{Eigenvalues and Eigenvectors}
  Consider a square matrix $A$. A real (or complex) number $\lambda$
  is an \alert{eigenvalue} if and only if there exists an \alert{eigenvector}
  $X\neq{}0$ such that
  \begin{equation}
    \label{eq:raeshaez}
    AX=\lambda{}X
  \end{equation}
  $AX=\lambda{}X$ is equivalent to the system of linear equations
  $(A-\lambda{}I)X=0$, which has a non-zero solution if and only if
  $A-\lambda{}I$ is singular,
\begin{equation}
  \label{eq:aeshopae}
  \det(A-\lambda{}I)=0
\end{equation}
$\det(A-\lambda{}I)$ is a polynomial in $\lambda$. It is called the
\alert{characteristic polynomial}.
\end{frame}

\begin{frame}
  \frametitle{Eigenvalues and Eigenvectors}
  \begin{block}{Characteristic Polynomial}
    The eigenvalues of a square matrix $A$ are the roots (solutions)
    of the polynomial equation $\det(A-\lambda{}I)=0$.
  \end{block}

  \bigskip

  {\ubung} Find the eigenvalues of
  \begin{equation}
    \label{eq:neevicuo}
    A=\left[
      \begin{array}{cc}
        3&-1\\
        2&0
      \end{array}\right]
  \end{equation}
and find one eigenvector for each eigenvalue.
\end{frame}

\begin{frame}
  \frametitle{Eigenvalues and Eigenvectors}
  Solution: find the determinant of
  \begin{equation}
    \label{eq:aeboapai}
    A-\lambda{}I=\left[
      \begin{array}{cc}
        3-\lambda&-1\\
        2&-\lambda
      \end{array}\right]
  \end{equation}
The characteristic polynomial is $\lambda^{2}-3\lambda+2$. The
eigenvalues of $A$ are $\lambda=2$ and $\lambda=1$. Now solve the
systems of linear equations for the eigenvectors:
\begin{equation}
  \label{eq:ahquaiwi}
  (A-2I)X=0\mbox{ for }\lambda=2
\end{equation}
and
\begin{equation}
  \label{eq:eetoodai}
  (A-I)X=0\mbox{ for }\lambda=1
\end{equation}
\end{frame}

\begin{frame}
  \frametitle{Eigenvalues and Eigenvectors}
  \begin{equation}
    \label{eq:jaegheed}
    A-2I=\left[
      \begin{array}{cc}
        1&-1\\
        2&-2
      \end{array}\right]\sim\left[
      \begin{array}{cc}
        1&-1\\
        0&0
      \end{array}\right]
  \end{equation}
  The solution set is
  \begin{equation}
    \label{eq:enahvodo}
S=\left\{X|X=s_{1}\left[
    \begin{array}{c}
      1\\
      1
    \end{array}\right],s_{1}\in\mathbb{R}\right\}
\end{equation}
$S$ is called the eigenspace of $\lambda=2$. All vectors except $X=0$
in the eigenspace of $\lambda$ are called eigenvectors belonging to
$\lambda$. Find the eigenspace of $\lambda=1$.
\end{frame}

\begin{frame}
  \frametitle{End of Lesson}
Next Lesson: Eigenvalues and Eigenvectors
\end{frame}

\end{document}
